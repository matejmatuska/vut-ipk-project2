\documentclass{article}

\usepackage[left=1.5cm,text={18cm, 24.5cm},top=2cm]{geometry}
\usepackage{times}

\usepackage[hidelinks, unicode]{hyperref}
\usepackage{graphicx}
\graphicspath{ {images/} }

\begin{document}
\begin{titlepage}
    \begin{center}
        \Huge
        \textsc{Vysoké učení technické v~Brně\\}
        \huge
        \textsc{Fakulta informačních technologií\\}
        \vspace{\stretch{0.382}}
        \LARGE

        IPK Project documentation\\
        Variant Zeta - Packet sniffer
        \vspace{\stretch{0.618}}
    \end{center}
{\Large 2022 \hfill Matej Matuška (xmatus36)}
\end{titlepage}

\tableofcontents
\pagebreak

\section{Introduction}
The goal of this project was to implement a simple packet sniffer program, which
can listen on different interfaces and capture incoming packets. Packets can be
filtered by port number and/or protocol.

\section{Design and implementation}
Program was implemented in C++, which offers easier string manipulation than C.
The source code is located in single file ipk-sniffer.cpp.
Capture of packets heavily realies on libpcap library. Parsing, formatting
and priting of various fields from protocol headers is then managed by the
program itself. Multiple system headers containing structures representing
protocol headers are also included:
\begin{itemize}
    \item net/ethernet.h
    \item netinet/ether.h
    \item netinet/in.h
    \item netinet/ip.h
    \item netinet/ip6.h
    \item netinet/tcp.h
    \item netinet/udp.h
\end{itemize}

\subsection{Notable funtions}
Program is structured into functions, representing different network protocols.
Functions representing lower layer protocols call funtions representing higher
level protocols. Because of this structure, program can be easily extended to
handle other protocols.

Function \texttt{process\_packet()} handles each packet and prints data relevant
to all packets such as timestamp, MAC adresses and raw bytes. Depending on
EtherType corresping function is called to further print more data.

Functions \texttt{process\_ip()} and \texttt{process\_ip6} then print IP and
IPv6 adresses, respectively. And then, depending on enclosed protocol call
corresponding function to print protocol data.

Functions \texttt{process\_tcp()} and \texttt{process\_udp()} print data
relevant for these protocols, such as port numbers.

\section{Example output}
Print one TCP packet from port 80, listen on interface wlp3s0:
\begin{verbatim}
> ./ipk-sniffer -i wlp3s0 --tcp -p 80
timestamp: 2022-04-24T14:27:59.204+02:00
src MAC: 50:5b:c2:a1:c2:17
dst MAC: 00:1a:1e:06:87:10
frame length: 74 bytes
src IP: 100.64.205.183
dst IP: 172.64.155.188
src port: 60028
dst port: 80
0x0000:  00 1a 1e 06 87 10 50 5b  c2 a1 c2 17 08 00 45 00 ......P[ ......E.
0x0010:  00 3c 20 e3 40 00 40 06  9f e4 64 40 cd b7 ac 40 .< .@.@. ..d@...@
0x0020:  9b bc ea 7c 00 50 b1 49  49 9e 00 00 00 00 a0 02 ...|.P.I I.......
0x0030:  fa f0 7a 23 00 00 02 04  05 b4 04 02 08 0a be c8 ..z#.... ........
0x0040:  9b 57 00 00 00 00 01 03  03 07 00 00 00 00 00 00 .W...... ........
\end{verbatim}

\section{Testing}
Program was tested intensively during implementation by comparing its outputs
with outputs from Wireshark, which is a well respected open source program.

\subsection{TCP packet test}
\begin{figure}[ht]
    \centering
    \includegraphics[scale=0.45]{wireshark-tcp.png}
    \caption{Outuput from WireShark}
\end{figure}
\begin{figure}[ht]
    \centering
    \includegraphics[scale=0.6]{sniffer-tcp.png}
    \caption{Outuput from ipk-sniffer}
\end{figure}

\subsection{UDP packet test}
\begin{figure}
    \centering
    \includegraphics[scale=0.4]{wireshark-udp.png}
    \caption{Outuput from WireShark}
\end{figure}
\begin{figure}
    \centering
    \includegraphics[scale=0.4]{sniffer-udp.png}
    \caption{Outuput from ipk-sniffer}
\end{figure}

\newpage

\subsection{ICMP packet test}
\begin{figure}[ht]
    \centering
    \includegraphics[scale=0.45]{wireshark-icmp.png}
    \caption{Outuput from WireShark}
\end{figure}
\begin{figure}[ht]
    \centering
    \includegraphics[scale=0.6]{sniffer-icmp.png}
    \caption{Outuput from ipk-sniffer}
\end{figure}

\newpage

\subsection{ARP packet test}
\begin{figure}[h]
    \centering
    \includegraphics[scale=0.45]{wireshark-arp.png}
    \caption{Outuput from WireShark}
\end{figure}
\begin{figure}[h]
    \centering
    \includegraphics[scale=0.6]{sniffer-arp.png}
    \caption{Outuput from ipk-sniffer}
\end{figure}

\newpage
\section{References}
Sources used while implementing this project:
\begin{itemize}
    \item Man page of pcap. Available at:\\
        \url{https://www.tcpdump.org/manpages/libpcap-1.8.1/pcap.3pcap.html}

    \item I'm trying to build an RFC3339 timestamp in C. How do I get the
        timezone offset?. \emph{Stack Overflow}. Available at:\\
        \url{https://stackoverflow.com/a/48772690}

    \item IPv4.\emph{Wikipedia}. Available at:\\
        \url{https://en.wikipedia.org/wiki/IPv4}

    \item IPv6. \emph{Wikipedia}. Available at:\\
        \url{https://en.wikipedia.org/wiki/IPv6}

    \item Ethernet frame. \emph{Wikipedia}. Available at:\\
        \url{https://en.wikipedia.org/wiki/Ethernet_frame}

    \item Address Resolution Protocol. \emph{Wikipedia}. Available at:\\
        \url{https://en.wikipedia.org/wiki/Address_Resolution_Protocol}

    \item Transmission Control Protocol. \emph{Wikipedia}. Available at:\\
        \url{https://en.wikipedia.org/wiki/Transmission_Control_Protocol}

    \item User Datagram Protocol. \emph{Wikipedia}. Available at:\\
        \url{https://en.wikipedia.org/wiki/User_Datagram_Protocol}

    \item Internet Control Message Protocol. \emph{Wikipedia}. Available at:\\
        \url{https://en.wikipedia.org/wiki/Internet_Control_Message_Protocol}

\end{itemize}

\end{document}
